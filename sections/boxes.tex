\section{Box extensions}

I've used \term{tcolorbox} to create a bunch of different boxes that can be used throughout the program. Look in the \term{boxes.tex} file, to see how they're defined and used.

\subsection{Sample ShowcaseCommand Demonstration}
Just for fun, I'll also showcase one of the commands that are used a lot throughout this document to show how commands are used.
\textcolor{red}{Not done yet.}
\subsection{Sample Syntax Demonstrations}
Here we will demonstrate how the syntax box works and is used. 

\subsubsection*{Without example usage:}
\showcaseCommand{newsyntax}{title,label,syntax file}{The title of the syntax box, the unique identifier for cross-referencing, the file path to the syntax definition file}

\newsyntax{Syntax for creating function pointer in C without usage}{syntax label}{box_extensions_crap/syntax/syntax}

\subsubsection*{With example usage:}
\showcaseCommand{newsyntax}{title, label, syntax file, output file, output lang}{The title of the syntax box, the unique identifier for cross-referencing, the file path to the syntax definition file, the file path to the output example file, the programming language used in the output example.}

\newsyntax{Syntax for creating function pointer in C with usage}{syntax label}{box_extensions_crap/syntax/syntax}{box_extensions_crap/syntax/output}{C}

\subsection{Sample Code Demonstrations}
These boxes allow you to present various types of code snippets, whether they produce an output or not. We've showcased this flexibility with C and Java code examples. This framework isn't limited to just these languages - if \term{lstlisting} doesn't include your desired language, you have the freedom to create your own set. It's a versatile way to enhance code visibility and comprehension.

\subsubsection{C and Java Code Example (No Output)}
\showcaseCommand{newlisting}{title, label, code language, code file}{The title of the code listing, the unique identifier for cross-referencing, the programming language of the code, the file path to the source code file}


\newlisting{A \term{C} code example \emph{without} output}{c example code}{C}{box_extensions_crap/code/program.c}

\newlisting{A \term{Java} code example \emph{without} output}{java example code}{Java}{box_extensions_crap/code/program.java}


\subsubsection{C Code Example (Includes Output)}
\showcaseCommand{newlisting}{title, label, code language, code file, example file}{The title of the code listing, the unique identifier for cross-referencing, the programming language of the code, the file path to the source code file, the file path to an example output or use of the code.}

\newlisting{A \term{C} code example \emph{with} output}{c example code}{C}{box_extensions_crap/code/program.c}{box_extensions_crap/code/output}

\subsection{Sample Lemma Demonstrations}
\showcaseCommand{mlenma}{title, label, content}{The title of the lemma, The unique identifier for cross-referencing, The content of the lemma}

\mlenma{Arithmetic Sum Lemma}{ArithSum}{
The sum $S$ of an arithmetic sequence with first term $a$, common difference $d$, and $n$ terms is given by: $S = \frac{n}{2} [2a + (n - 1)d]$.
\\ \\
Proof:
Adding the sequence $a, a+d, \ldots, a+(n-1)d$ to its reverse yields $2S = n [2a + (n-1)d]$. Thus, $S = \frac{n}{2} [2a + (n - 1)d]$. QED.
}
\subsection{Sample Definition Demonstrations}
\showcaseCommand{newdfn}{title, label, content}{The title of the definition, The unique identifier for cross-referencing, The detailed explanation or description}
\newdfn{VSpace}{Vector Space}{%
A vector space $(V, F)$ over a field $F$ (usually $\mathbb{R}$ or $\mathbb{C}$) is a set $V$ along with two operations, vector addition and scalar multiplication, satisfying the following axioms:\\
1. $\forall \mathbf{u}, \mathbf{v} \in V$, $\mathbf{u} + \mathbf{v} \in V$.\\
2. $\forall \mathbf{u}, \mathbf{v}, \mathbf{w} \in V$, $(\mathbf{u} + \mathbf{v}) + \mathbf{w} = \mathbf{u} + (\mathbf{v} + \mathbf{w})$.\\
3. $\exists \mathbf{0} \in V$ such that $\forall \mathbf{u} \in V$, $\mathbf{u} + \mathbf{0} = \mathbf{u}$.\\
4. $\forall \mathbf{u} \in V$, $\exists \mathbf{v} \in V$ such that $\mathbf{u} + \mathbf{v} = \mathbf{0}$.\\
5. $\forall \mathbf{u} \in V$ and $\forall a, b \in F$, $a \cdot (b \cdot \mathbf{u}) = (a \cdot b) \cdot \mathbf{u}$.\\
6. $\forall \mathbf{u} \in V$ and $\forall a \in F$, $a \cdot \mathbf{u} \in V$.
}


\subsection{Sample Table Demonstrations}


\subsection{Making References to the Boxes}
Soon to come... When I am not too lazy...