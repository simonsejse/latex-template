\section{Math extensions}

% Extra commands just for the purposes of this document.
\def\commandstyle{{\ttfamily\textbackslash}}
\def\command#1{\texttt{\textbackslash #1}}

The math exetensions were inspired as I was making my way through
\cite{concrete-mathematics}, and various courses at DIKU. It seems to me that
these macros are useful in general.

\subsection{Math: Common Shortcuts}

The commands below are typically used to save time when typesetting common mathematical structures.

\vspace{0.1in}

\noindent
\begin{tabular}{lcl}

\textbackslash mbb\{symbol\} & $\mbb{\mathtt{symbol}}$ & (for \emph{blackboard bold} font) \\
\textbackslash mfk\{symbol\} & $\mfk{\mathtt{symbol}}$ & (for \emph{mathematical fraktur} font) \\
\textbackslash N & $\N$ & (as in, \emph{natural numbers}) \\
\textbackslash C & $\C$ & (as in, \emph{complex numbers}) \\
\textbackslash R & $\R$ & (as in, \emph{real numbers}) \\
\textbackslash pr & $\pr$ & (probability) \\
\textbackslash Q & $\Q$ & (as in, \emph{rational numbers}) \\
\textbackslash Z & $\Z$ & (as in, \emph{integers}) \\
\textbackslash la & $\la$ & (leftwards arrow) \\
\textbackslash La & $\La$ & (double leftwards arrow) \\
\textbackslash ra & $\ra$ & (rightwards arrow) \\
\textbackslash Ra & $\Ra$ & (double rightwards arrow) \\
\textbackslash lp & $\lp\r.$ & (open parenthesis) \\
\textbackslash rp & $\l.\rp$ & (close parenthesis) \\
\textbackslash lk & $\lk\r.$ & (open bracket) \\
\textbackslash rk & $\l.\rk$ & (close bracket) \\
\textbackslash ll & $\ll\r.$ & (open line) \\
\textbackslash rl & $\l.\rl$ & (close line) \\
\textbackslash lb & $\lb\r.$ & (open brace) \\
\textbackslash rb & $\l.\rb$ & (close brace) \\

\end{tabular}

\vspace{0.1in}

\subsection{Math: Sometimes Helpful Macros}

These macros are particularly useful when dealing with more complex mathematical structures or constructs.

\vspace{0.1in}

\noindent
\begin{tabular}{lcl}

\textbackslash vfunc\{name\}\{domain\}\{codomain\}\{element\}\{image\} & $\vfunc{\mathtt{name}}{\mathtt{domain}}{\mathtt{codomain}}{\mathtt{element}}{\mathtt{image}}$ & (a \emph{verbose function} declaration) \\
\textbackslash pdif\{function\}\{variable\}\{order\} & $\pdif{\mathtt{function}}{\mathtt{variable}}{\mathtt{order}}$ & (partial differentiation) \\
\textbackslash del & $\del$ & (nabla or del operator) \\
\textbackslash eps & $\eps$ & (epsilon) \\
\textbackslash inv & $\inv$ & (inverse) \\
\textbackslash pa & $\pa$ & (partial derivative) \\
\textbackslash vp & $\vp$ & (phi) \\
\textbackslash y & $\y$ & (infinity) \\
\textbackslash th & $\th$ & (theta) \\

\end{tabular}

\subsection{Groups}

Very often, mathematical expressions make use of grouping constructs such as
$\ceil{}$, $\floor{}$, $\p{}$, etc. These constructs are relatively easy to use
in \LaTeX (with the \mono{amsmath} package), despite the fact that one has to
often distinguish between the left and right connectives, as with e.g.
\command{lfoor} and \command{rfloor}. What makes these groups particularly
impractical however, is that the height of the connectives is not automatically
adjusted to the content they enclose. To this end, one may resort to using the
commands \command{left} and \command{right}, as respective connective
prefixes\dots Yuk!  This lead to the specification of the following macros:

\vspace{0.1in}

\noindent
\begin{tabular}{lcl}

\command{ceil\{group\}} & $\ceil{\mathtt{group}}$ \\

\command{floor\{group\}} & $\floor{\mathtt{group}}$ \\

\command{set\{group\}} & $\set{\mathtt{group}}$ \\

\command{seq\{group\}} & $\seq{\mathtt{group}}$ & (as in, \emph{sequence}) \\

\command{card\{group\}} & $\card{\mathtt{group}}$ & (as in, \emph{cardinality})
\\

\command{chev\{group\}} & $\chev{\mathtt{group}}$ & (as in, \emph{chevrons}) \\

\command{p\{group\}} & $\p{\mathtt{group}}$ & (as in, \emph{parenstheses}) \\

\command{st\{group\}} & $\st{\mathtt{group}}$ & (as in, \emph{such that})

\end{tabular}

\subsection{Backus-Naur Form}

\vspace{0.1in}

\noindent
\begin{tabular}{lcl}

\command{nonterm\{group\}} & $\nonterm{group}$ \\

\command{term\{group\}} & $\term{group}$

\end{tabular}

\subsection{Cormen}

\begin{codebox}
\Procname{$\proc{Merge-Sort}(A,p,r)$}
\li \If $p < r$ \Then
\li $q \gets \floor{(p + r) / 2}$
\li $\proc{Merge-Sort}(A, p, q)$
\li $\proc{Merge-Sort}(A, q + 1, r)$
\li $\proc{Merge}(A, p, q, r)$
\End
\end{codebox}
